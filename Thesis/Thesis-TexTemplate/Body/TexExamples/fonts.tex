\chapter{��������ʾ��}
\section{�ı�ģʽ������}
These commands are used like `$\backslash$textit{italics text}'. The corresponding command in parenthesis is the "declaration form", which takes no arguments.
    The scope of the declaration form lasts until the next type style command or the end of the current group.
   The declaration forms are cumulative; i.e., you can say `$\backslash$ sffamily$\backslash$bfseries' to get sans serif boldface.
   You can also use the environment form of the declaration forms; e.g.`$\backslash$ begin{ttfamily}...$\backslash$end{ttfamily}'.

\begin{enumerate}
    \item {Roman $\backslash$ textrm ($\backslash$ rmfamily)}\\
    \textrm{\abc \\ \upabc}
    \item {$\backslash$ textit ($\backslash$ itshape)}\\
    \textit{\abc \\ \upabc}
    \item {Emphasis $\backslash$emph }\\
    \emph{\abc \\ \upabc}
    \item {Boldface$\backslash$ textbf ($\backslash$ bfseries)}\\
    \textbf{\abc \\ \upabc}
    \item {Slanted $\backslash$ textsl ($\backslash$ slshape)}\\
    \textsl{\abc \\ \upabc}
    \item {Sans serif $\backslash$ textsf ($\backslash$ sffamily)}\\
    \textsf{\abc \\ \upabc}
    \item {Small caps $\backslash$ textsc ($\backslash$ scshape)}\\
    \textsc{\abc \\ \upabc}
    \item {Typewriter $\backslash$ texttt ($\backslash$ ttfamily)}\\
    \texttt{\abc \\ \upabc}
    \item {Main document font $\backslash$ textnormal ($\backslash$ normalfont)}\\
    \textnormal{\abc \\ \upabc}
\end{enumerate}

\section{��ѧģʽ������}
\begin{enumerate}
    \item {Roman, for use in math mode $\backslash$ mathrm }\\
    $\mathrm{\abc}$\\
    $\mathrm{\upabc}$
    \item {Boldface, for use in math mode $\backslash$ mathbf }\\
    $\mathbf{\abc}$\\
    $\mathbf{\upabc}$
    \item {Sans serif, for use in math mode $\backslash$ mathsf }\\
    $\mathsf{\abc}$\\
    $\mathsf{\upabc}$
    \item {Typewriter, for use in math mode $\backslash$ mathtt }\\
    $\mathtt{\abc}$\\
    $\mathtt{\upabc}$
    \item {For use in math mode, e.g. inside another type style declaration $\backslash$ mathnormal }\\
    ${\abc}$\\
    ${\upabc}$
    \item { `Calligraphic' letters, for use in math mode $\backslash$ mathcal }\\
    $\mathcal{\upabc}$
    \item {$\backslash$ mathbb }\\
    $\mathbb{\upabc}$
\end{enumerate}
\begin{verbatim}
   In addition, the command `\mathversion{bold}' can be used for switching
   to bold letters and symbols in formulas. `\mathversion{normal}' restores
    the default.
\end{verbatim}

\section{�������ڱ�ʾ����}

\begin{verbatim}
\CJKtoday\today\end{verbatim}

\CJKtoday \today

\begin{verbatim}
\CJKtoday[0]\today\end{verbatim}

\CJKtoday[0] \today

\begin{verbatim}
\CJKtoday[1]\today\end{verbatim}

\CJKtoday[1] \today

\begin{verbatim}
\CJKtoday[2]\today\end{verbatim}

\CJKtoday[2] \today
